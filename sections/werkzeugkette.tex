\section{Werkzeugkette und Versionskontrolle}
	
\subsection{Befehlskette}
	
git <Kommando>:

\begin{itemize}
\item \textbf{help:} Liste der Kommandos
\item \textbf{init:} Depot anlegen
\item \textbf{clone:} Depot eines Projekts laden
\item \textbf{add:} Neue Datei hinzufügen und in Staging Area übernehmen
\item \textbf{commit:} Änderungen in das eigene Depot übernehmen
\item \textbf{push:} Commits in ein anderes Depot übertragen
\item \textbf{fetch:} Änderungen von einem anderen Depot holen
\item \textbf{merge:} Änderungen von einem Branch in einen anderen übertragen
\item \textbf{pull:} Ausführen von fetch und merge
\end{itemize}
	
\subsection{Interaktion innerhalb Local- und Remote Repository}
	
\begin{center}
\includegraphics[width=0.57\textwidth]{../images/gitInteraktion.png}
\end{center}
	
\subsection{Unterschied: GIT und SVN}
	
\begin{center}
\begin{tabular}{c|c}
\textbf{GIT} & \textbf{SVN} \\
\hline
- Depot ist für jeden User \textbf{lokal} & - Depot wird \textbf{auf einem Server} abgelegt \\
- Operationen \textbf{offline ausführbar} & - \textbf{Optimistisches Ausbuchen} \\
- \textbf{Kryptografische Sicherung} der Historie & - Versioniert gesamtes Depot \\
- Speichert \textbf{Schnappschüsse} & - Speichert \textbf{Deltas} \\
\end{tabular}
\end{center}
	
\subsection{Versionskontrolle}
	
\subsubsection{Vorwärtsdelta}

\textbf{Anfangsversion als Ausgangspunkt} und alle Deltas danach werden gespeichert.
	
\subsubsection{Rückwärtsdelta}
	
\textbf{Neueste Version als Ausgangspunkt} und alle Deltas davor werden gespeichert.
	
\begin{tabular}{c|c|c}
& \textcolor{green}{\textbf{+}} & \textcolor{red}{\textbf{-}} \\
\hline
Vorwärtsdelta & Schneller Zugriff auf alte Version & Langsamer Zugriff auf alte Version \\
\hline
Rückwärtsdelta & Schneller Zugriff auf alte Version & Langsamer Zugriff auf neue Version \\
\end{tabular}
	
\subsubsection{Ein- und Ausbuchen}

\begin{itemize}
\item Optimistisch: \textbf{Mehrfaches} Ein- und Ausbuchen \textbf{ohne Änderungsreservierung}
\item Strikt: Ein- und Ausbuchen \textbf{mit Änderungsreservierung}
\end{itemize}